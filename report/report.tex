\documentclass{article}
\title{Senior Projects Holography Report}
\date{2018-12-02}
\author{Ben Sterling, Dongkyu Kim, Junbum Kim, David Kim}

\usepackage{amsmath}
\usepackage{amssymb}
\usepackage{mathrsfs}
\usepackage{mathtools}
\usepackage[pdftex]{graphicx}

\begin{document}
\maketitle
\clearpage

\section{Introduction}

The objective of the project is to create a 3D sample of a real-life object and project onto a surface as a hologram. The applications of this project include displaying an artwork or a building schematic in 3D to help visualize the product for artists and architects. We believe this is a great alternative to existing solutions as the previous technologies are difficult and time consuming. Several prototyping tools and techniques exist including AutoCAD, SolidWorks, and 3D printing. The AutoCAD and SolidWorks are programs that require the user to manually model the real life object, and 3D printing also requires a model that has been already modeled digitally. In addition, 3D printing often takes a while to create a 3D object in real life depending on the size of the object. The substages of the projects can be largely divided into include the sampling stage, digital conversion stage, digital projection stage, and optical holography projection stage. The following report discusses each step in more detail.

\section{Sampling Stage}

As the focus of the project was to investigate holographic application, existing solutions for 3D reconstruction tools were used for the project. Xbox Kinect for Windows Development was the specific machine used for the project. Windows Presentation Foundation (WPF) application was built to communicate the hardware with software. Kinect Studio Version 2.0 and Kinect Software Development Kits were installed to develop the program. Specifically, the KinectFusionExplorer module proved practical for the purpose. The module allows 3D modelling from a single photo shoot from one angle by reconstructing picture from plain camera, infrared sensors, and depth sensors. Although the technology cannot provide a 360-degree capture of the sample, it does reconstruct the sides of the image with only small depth errors. The rotation of the model allows users to see the sides of the real object within the computer. Note that USB 3.0 is required to reproduce the program, as the Kinect hardware requires the extra bandwidth for communication of data. 

\section{Digital Conversion Stage}

Bitmaps can be constructed from the 3D model constructed, but the exact input for source light modulator has not exactly been specified yet, and since the development of the digital conversion stage basically depends on the exact requirement of SLM, this field has yet not been developed properly yet.

\section{Digital Projection Stage}

SLM is simply speaking, a monitor. It takes a digital output from servers via an HDMI cable. Python programs will run on top of the WPF application to project data onto SLM via wireless HDMI. Modules such as slmpy exist to perform SLM projection. This is yet a step being investigated.

\section{Holography Background}

\subsection{Wave Fundamentals}

The fundamentals of Holography start from Maxwell's Equations. In the most general case, they are formulated as follows:

\begin{equation}
	\nabla \cdot \vec{D} = \rho\\
\end{equation}

\begin{equation}
	\nabla \cdot \vec{B} = 0
\end{equation}

\begin{equation}
	\nabla \times \vec{H} = \vec{J} + \frac{\partial \vec{D}}{\partial t}
\end{equation}

\begin{equation}
	\nabla \times \vec{E} = -\frac{\partial \vec{B}}{\partial t}
\end{equation}

However, we are interested in utilizing these equations to study optics. Since light waves travel in the air, \(\rho\) = 0 and \(\vec{J}\) = \(\vec{0}\) because there is no source charge or current in the air. Here are the equations we obtain setting sources to zero:

\begin{equation}
	\nabla \cdot \vec{D} = 0
\end{equation}

\begin{equation}
	\nabla \cdot \vec{B} = 0
\end{equation}

\begin{equation}
	\nabla \times \vec{H} = \frac{\partial \vec{D}}{\partial t}
\end{equation}

\begin{equation}
	\nabla \times \vec{E} = -\frac{\partial \vec{B}}{\partial t}
\end{equation}

Lastly, we want to relate D to E, and B to H. The most general relationships that describes these quantities are:

\begin{equation}
	\vec{D} = \epsilon_{0}\vec{E} + \vec{P}
\end{equation}

\begin{equation}
	\vec{B} = \mu_{0}(\vec{H} + \vec{M})
\end{equation}

We assume that all the materials we are working in are Linear, Homogeneous, and Isotropic. The last two equations can be simplified to:

\begin{equation}
	\vec{D} = \epsilon\vec{E}
\end{equation}

\begin{equation}
	\vec{B} = \mu\vec{H}
\end{equation}

where \(\epsilon\) = \(\epsilon_{r}\)\(\epsilon_{0}\) and \(\mu\) = \(\mu_{r}\)\(\mu_{0}\) are constants. If we combine the last two equations along with Maxwell's Equations we get:

\begin{equation}
	\nabla \cdot \vec{E} = 0
\end{equation}

\begin{equation}
	\nabla \cdot \vec{H} = 0
\end{equation}

\begin{equation}
	\nabla \times \vec{H} = \epsilon\frac{\partial E}{\partial t}
\end{equation}

\begin{equation}
	\nabla \times \vec{E} = -\mu\frac{\partial H}{\partial t}
\end{equation}

then we utilize the following identity from vector calculus:

\begin{equation}
	\nabla^2F = \nabla(\nabla \cdot F) - \nabla\times\nabla\times F
\end{equation}

and get:

\begin{equation}
	\nabla^2E - \mu\epsilon\frac{\partial^2 E}{\partial t^2} = 0
\end{equation}

\begin{equation}
	\nabla^2H - \mu\epsilon\frac{\partial^2 H}{\partial t^2} = 0
\end{equation}

The Multi-Dimensional Fourier Transform is defined as follows:

\begin{equation}
	\hat{f} (\vec{k}) = \iint \limits_{R^n}^{} f(\vec{x})e^{-j \vec{k} \cdot \vec{x}} \,d\vec{x}
\end{equation}

and a plane wave is defined as follows:

\begin{equation}
	\vec{E} = \vec{E_{0}}e^{j \omega t - j\vec{k} \cdot \vec{x}} 
\end{equation}

Because E and H field are functions of x, y, z, and time, it is possible to span all E and H fields with a superposition of plane waves. This motivates the notation:

\begin{equation}
	\vec{E} = \vec{E_{0}}\psi
\end{equation}

where

\begin{equation}
	\psi = e^{j \omega t - j\vec{k} \cdot \vec{x}}
\end{equation}

Thus, in \(\psi\) notation, the wave equation turns into:

\begin{equation}
	\nabla^2\psi - \mu\epsilon\frac{\partial^2\psi}{\partial t^2} = 0
\end{equation}

\subsection{Plane and Spherical Wave Applications}

For a single frequency, we are interested in the complex amplitude, \(\psi_{p}\) we define as follows:

\begin{equation}
	\psi = \psi_{p}e^{j \omega t}
\end{equation}

We can think of this quantity like the phasor representation of \(\psi\).

The central application of the classical theory is to develop the math behind Fresnel Diffraction. An aperture can be visualized by this diagram:
\begin{center}
\includegraphics[width=100mm]{tupac5.png}
\end{center}

We also define \(k_{0}\) as the wave number in free space, such that

\begin{equation}
	\nabla^2\psi_{p} + k_{0}^2\psi_{p} = 0
\end{equation}

since

\begin{equation}
	\lambda f = c \rightarrow k_{0} = \omega/c
\end{equation}

By considering the incoming plane wave as a superposition of plane waves (like we did with the Multi-Dimensional Fourier Transform), we obtain the result:

\begin{equation}
	\mathscr{F} \Big\{ \frac{\partial^2 \psi_{p}}{\partial x^2} \Big\} = (-jk_{x})^2\Psi_{p} = -k_{x}^2\Psi_{p}
\end{equation}

and

\begin{equation}
	\mathscr{F} \Big\{ \frac{\partial^2 \psi_{p}}{\partial y^2} \Big\} = (-jk_{y})^2\Psi_{p} = -k_{y}^2\Psi_{p}
\end{equation}

where 

\begin{equation}
	\Psi_{p} = \mathscr{F} \{\psi_{p}\}
\end{equation}

Finally, if we take the Fourier Transform of both sides of (26) we obtain:

\begin{equation}
	\frac{\partial^2\Psi_{p}}{\partial z^2} + k_{0}^{2} \bigg ( 1 - \frac{k_{x}^2}{k_{0}^2} - \frac{k_{y}^2}{k_{0}^2} \bigg ) \Psi_{p} = 0
\end{equation}

While this looks complicated, it is really only an Ordinary Differential Equation of \(\Psi_{p}\) with respect to z. It's solution is:

\begin{equation}
	\Psi_{p} = \big(\Psi_{p}\vert_{z = 0}\big) exp\Bigg(-jk_{0}z\sqrt{1 - \frac{k_{x}^2}{k_{0}^2} - \frac{k_{y}^2}{k_{0}^2}}\Bigg)
\end{equation}

If we let

\begin{equation}
	\Psi_{p0} = \big(\Psi_{p}\vert_{z = 0}\big)
\end{equation}

Then we can recognize the solution in (32) as a transfer function. We define:

\begin{equation}
	\mathscr{H} = \frac{\Psi_{p}}{\Psi_{p0}} = exp\Bigg(-jk_{0}z\sqrt{1 - \frac{k_{x}^2}{k_{0}^2} - \frac{k_{y}^2}{k_{0}^2}}\Bigg)
\end{equation}

\(\mathscr{H}\) is called the Spatial Frequency Transfer Function of Propagation. It applies to our specific apparatus under consideration and describes Fresnel Diffraction of the output of a plane wave hitting an aperture. \(\psi_{p0}\) is the function controlled by the shape of the aperture; if this equals \(\delta(x,y)\), for example, then the output of the aperture is purely the Spatial Frequency Transfer Function response, because this is equivalent to convolving with a delta in 2-Dimensions, returning \(\mathscr{F}^{-1} \{\mathscr{H}\} \) in the time domain.

\subsubsection{Fresnel Diffraction}

We have \(\mathscr{H}\), but more often than not, the traveling waves make very small angles from the normal direction of the aperture. This allows us to utilize the Paraxial Approximation:

\begin{equation}
	\sqrt{1 - \frac{k_{x}^2}{k_{0}^2} - \frac{k_{y}^2}{k_0^2}} \approx 1 - \frac{k_{x}^2}{2k_{0}^2} - \frac{k_{y}^2}{2k_{0}^2}
\end{equation}

Once we apply this approximation, the Spatial Frequency Transfer Function of Propagation simplifies:

\begin{equation}
	\mathscr{H} = exp(-jk_{0}z)exp\bigg[\frac{j(k_{x}^2 + k_{y}^2)z}{2k_{0}}\bigg]
\end{equation}

In this specific example, we are using the Two Dimensional Fourier Transform (z remains constant):

\begin{equation}
	\Psi_{p}(k_{x},k_{y},z) = \iint \limits_{R^2}^{} \psi_{p}(x,y,z)e^{-jk_{x}x - jk_{y}} \,dx\,dy
\end{equation}

Thus we get the Spatial Impulse Response, h:

\begin{equation}
	h(x,y,z) = \mathscr{F}^{-1}\{\mathscr{H}\} = \frac{jk_{0}}{2\pi z} exp(-jk_{0}z)exp\bigg[\frac{-jk_{0}(x^2 + y^2)}{2z}\bigg]
\end{equation}

\subsection{Holography Fundamentals}

The simplest holography setup, and a good starting point to understanding the subject, is this collection of Fresnel Plates:

\begin{center}
	\includegraphics[width=100mm]{tupac7.png}	
\end{center}

First, we use a collimated laser so that the direction of propagation of each wave is parallel.
The first beam splitter (BS1) divides the laser in two paths of equal power, one will be used as a reference and the other contains the object information.
We recombine the waves at the second beam splitter (BS2). If the object wave is denoted as \(\psi_{0}\) and the reference wave is \(\psi_{r}\).
Since our aperture is a pinhole, our input function (referencing the Spatial Frequency Transfer Function) is \(\delta(x,y)\), thus:

\begin{equation}
	\begin{multlined}
		\psi_{0}(x,y,z_{0}) = \delta(x,y) * h(x,y,z_{0}) = h(x,y,z_{0}) 
		\\= exp(-jk_{0}z_{0})\frac{jk_{0}}{2\pi z_{0}}exp\bigg[\frac{-jk_{0}(x^2 + y^2)^2}{2z_{0}}\bigg]
	\end{multlined}
\end{equation}

and \(\psi_{r}\) is just a plane wave propagating in the \(z_{0}\) direction:

\begin{equation}
	\psi_{r} = a\,exp(-jk_{0}z_{0})
\end{equation}

It is important to note that the light intensity is proportional to square of the complex amplitude, I(x,y) \(\propto\) \(|\psi_{p}(x,y)|^2\).
Thus we define transmittance, t(x,y) as:

\begin{equation}
	t(x,y) = |\psi_{p}(x,y)|^2
\end{equation}

For our specific case:

\begin{equation}
	\begin{multlined}
	t(x,y) = |\psi_{o}(x,y) + \psi_{r}(x,y)|^2
	=(\psi_{o} + \psi_{r})(\psi_{o} + \psi_{r})^*
	\\= a^2 + (\frac{k_{0}^2}{2\pi z_{0}})^2 - \frac{-jk_{0}a}{2\pi z_{0}}
	exp\bigg[ \frac{jk_{0}}{2z_{0}}(x^2 + y^2) \bigg] + 
	\\\frac{-jk_{0}a}{2\pi z_{0}}
	\exp\bigg[\frac{-jk_{0}}{2z_{0}}(x^2 + y^2) \bigg]
	\end{multlined}
\end{equation}

We can recognize the sine function inside the last line:

\begin{equation}
	t(x,y) = a^2 + \big(\frac{k_{0}}{2\pi z_{0}}\big)^2 + 
	\frac{ak_{0}}{\pi z_{0}}sin\bigg[\frac{k_{0}}{2z_{0}}(x^2 + y^2)\bigg]
\end{equation}

And thus, we can observe the diffraction pattern of this aperture as follows:

\begin{center}
\includegraphics[width=50mm]{tupac8.png}
\end{center}

By inspection, the output of the machine obeys a sinusoidal pattern, with some DC value in towards the center (x = 0 and y = 0), and the frequency
of the sinusoid increases parabolically moving radially outward.
Also, for any tangent vector \(v_{p}\), the spatial frequency is defined as the directional derivative of the angle:

\begin{equation}
	v_{p}\bigg[ \frac{k_{0}(x^2 + y^2)}{2z_{0}} \bigg] = \frac{k_{0}}{z_{0}}(v_{1}p_{1} + v_{2}p_{2}), \forall v_{p} \in T_{p} \mathbb{R}^2
\end{equation}

This result is more significant in differential form:

\begin{equation}
	d \bigg( \frac{k_{0}(x^2 + y^2)}{2z_{0}} \bigg) = \frac{k_{0}}{z_{0}}(xdx + ydy)
\end{equation}

The frequency increases linearly in any direction of our choosing.
It is important to note that an observer at a distance will see the following 
\(\psi_{rec}\) instead because they are at a distance, z, away from the
recording medium.

\begin{equation}
	\psi_{p} = \psi_{rec}t(x,y)*h(x,y,z)
\end{equation}

Where \(\psi_{rec}\) is the field on the projecting surface.

\subsubsection{3D Holographic Imaging}

In this next section, we wish to identify the effects of different coplanar displacements 
(x and y) and different distances from the medium (z). We can visualize this problem with the following diagrams:

\begin{center}
\includegraphics[width=100mm]{tupac11.png}
\end{center}

\begin{center}
\includegraphics[width=100mm]{tupac12.png}
\end{center}

The first graph displays two source points with a reference wave R, and the second displays the reconstructed point r (the location a viewer SEE'S the object).

The following identity under convolution helps to simplify the upcoming calculations.

\begin{equation}
	\delta(x - \alpha,y - \beta)*h(x,y,z) = \delta(x,y)*h(x - \alpha,y - \beta,z)
\end{equation}

The physical significance of this identity is that being off-center from the pinhole
aperture is equivalent to translating the center Spatial Impulse Response: a very important property.

Thus, we can find the three relevant complex fields as follows:

\begin{equation}
	\psi_{p1} = \delta(x - \frac{h}{2},y)*h(x,y,R) = h(x - \frac{h}{2},y,R)
\end{equation}

\begin{equation}
	\psi_{p2} = \delta(x + \frac{h}{2},y)*h(x,y,R + d) = 
	h(x - \frac{h}{2},y,R + d)
\end{equation}

\begin{equation}
	\psi_{pR} = \delta(x + a,y)*h(x,y,l_{1}) = h(x + a,y,l_{1})
\end{equation}

Now, the intensity can be found as follows (by definition)

\begin{equation}
	\begin{multlined}
	t(x,y) = |\psi_{p1} + \psi_{p2} + \psi_{pR}|^2
	\\ (\psi_{p1} + \psi_{p2} + \psi_{pR})(\psi_{p1} + \psi_{p2} + \psi_{pR})^*
	\end{multlined}
\end{equation}

Each \(\psi_{pi}\psi_{pi}^*\) for \(i \in \mathbb{Z}^+\) is not very interesting, because the wave parts cancel and they are just zeroth order waves.
Also, note \(\psi_{pi}\psi_{pj}^* = (\psi_{pj}\psi_{pi}^*)^* \), so we only need three transmittances to determine the intensity

If we take one transmittance:

\begin{equation}
	t_{rel1}(x,y) = \psi_{p1}^*\psi_{pR}(x,y) = exp(\frac{jk_{0}}{2R}[(x - \frac{h}{2})^2 + y^2])exp(\frac{-jk_{0}}{2R}[(x + a)^2 + y^2])
\end{equation}

and then solve for the reflected \(\psi\), we get:

\begin{equation}
	\psi_{pr}(x,y)t_{rel1}(x,y)*h(x,y,z)
\end{equation}

Where:
\begin{equation}
	\psi_{pr} = \delta(x - b,y)*h(x,y,l_{2}) = h(x - b,y,l_{2})
\end{equation}

This must be true because INDEPENDENT of the points 1 and 2, the field is being
affected by a plane wave coming from the reconstruction point. Thus, if we want to 
approximate the reflected wave, it is given by:

\begin{equation}
	\begin{multlined}
	\psi_{ref} \propto exp\bigg(\frac{-jk_{0}}{2l_{2}}[(x - b)^2 + y^2]\bigg)
	exp\bigg(\frac{-jk_{0}}{2R}[(x - \frac{h}{2})^2 + y^2]\bigg)
	\\exp\bigg(\frac{-jk_{0}}{2l_{1}}[(x + a)^2 + y^2]\bigg)
	*\frac{jk_{0}}{2\pi z}exp\bigg(\frac{-jk_{0}}{2z}[x^2 + y^2]\bigg)
	\end{multlined}
\end{equation}

At this point in the derivation, we must remember the definition of 2D convolution we are working with:

\begin{equation}
	\big(f_{1}*f_{2}\big)(x,y) = \iint \limits_{R^2}^{} f_{1}(x',y')f_{2}(x - x',y - y') \,dx'\,dy'
\end{equation}

By far the most confusing part of this derivation: the coefficients in front of \(x'^2\) and \(y'^2\) must equal zero because the convolution in (55) must evaluate to a \(\delta\) function.
Think of it this way: If it did not evaluate to a \(\delta\), this would imply the field has multiple non-zero space values for a single image point; we do not allow a single point 
to spread into a cloud, this would violate our empirical observations.

Thus by inspection, we derive the equation:

\begin{equation}
	\frac{1}{R} - \frac{1}{l_{1}} - \frac{1}{l_{2}} - \frac{1}{z_{r1}} = 0
\end{equation}

Through very tedious math, we get that:

\begin{equation}
	\psi_{ref} \propto \delta \bigg[ x + z_{r1}\bigg( 
	\frac{b}{l_{2}} - \frac{h}{2R} - \frac{a}{l_{1}}\bigg),y\bigg]
\end{equation}

Similarly for the second point:

\begin{equation}
	\frac{1}{R + d} - \frac{1}{l_{1}} - \frac{1}{l_{2}} - \frac{1}{z_{r2}} = 0
\end{equation}

\begin{equation}
        \psi_{ref} \propto \delta \bigg[ x + z_{r2}\bigg(
        \frac{b}{l_{2}} - \frac{h}{2(R + d)} - \frac{a}{l_{1}}\bigg),y\bigg]
\end{equation}

These equations are the starting point to understanding magnification and translation distortion of images.
For the most part, we are simplifying the problem of displaying an image to 
effects between two points.

\subsubsection{Holographic Magnification}

In this context, magnification is defined as:

\begin{equation}
	M_{Long}^r = \frac{z_{r2} - z_{r1}}{d}
\end{equation}

This is literally a ratio of the hologram depth over the real depth between these two points.

We substitute from equation (57) and (59) (Assuming: R \(\gg\) d)

\begin{equation}
	M_{Long}^r = \bigg(\frac{l_{1}l_{2}}{l_{1}l_{2} - Rl_{1} - Rl_{2}}\bigg)^2
\end{equation}

We define lateral magnification as the difference in x virtual positions over the real x distance (h). We use equations (58) and (60):

\begin{equation}
	\begin{multlined}
		M_{Lat}^r = \frac{z_{r2}\bigg( \frac{b}{l_{2}} + \frac{h}{2(R + d)} - \frac{a}{l_{1}}\bigg) - z_{r1}\bigg(\frac{b}{l_{2}} - \frac{h}{2R} - \frac{a}{l_{1}} \bigg)}{h}
		\\\approx \frac{(z_{r2} - z_{r1})\bigg(\frac{b}{l_{2}} -\frac{a}{l_{1}} \bigg) + (z_{r2} + z_{r1})\frac{h}{2R}}{h}
	\end{multlined}
\end{equation}

assuming \(R \gg d\)

If we align the points in a special way such that:

\begin{equation}
	\frac{b}{l_{2}} - \frac{a}{l_{1}} = 0
\end{equation}

We get the fascinating result that

\begin{equation}
	M_{Long}^r = (M_{Lat}^r)^2
\end{equation}

\subsubsection{Holographic Translation}

The equation for Lateral Magnification gives us the distortion between two holographic points if (64) is not met:

\begin{equation}
	\Delta x = x_{1} - x_{2} = (z_{r2} - z_{r1})\bigg(\frac{b}{l_{2}} - \frac{a}{l_{1}}\bigg)
\end{equation}

This figure shows how an artificial \(\Delta x\) comes about from erroneous orientation

\begin{center}
\includegraphics[width=100mm]{tupac13.png}
\end{center}

\subsubsection{Off-Axis Holography}

The standard holography setup we have studied is called on-axis holography.
The only problem with this approach is that we get twin images when we expand
the full equation for transmittance: (the conjugation terms make the twin images)

\begin{equation}
	t(x,y) = |\psi_{r} + \psi_{o}|^2 = |\psi_{r}|^2 + |\psi_{o}|^2
	 + \psi_{r}^*\psi_{o} + \psi_{r}\psi_{o}^*
\end{equation}

Off-Axis holography takes care of this problem by introducing the reference
light at an angle \(\theta\)

\begin{center}
\includegraphics[width=100mm]{tupac9.png}
\end{center}

We can redo the logic of equation (67) as follows (the extra phase factor comes
out from the extra distance the reference wave needs to travel)

\begin{equation}
	\begin{multlined}
	t(x,y) = |\psi_{r}e^{jk_{0}sin\theta x} + \psi_{o}|^2 = |\psi_{r}|^2 + |\psi_{o}|^2
         + \psi_{r}^*\psi_{o}e^{-jk_{0}sin(\theta) x} + \psi_{r}\psi_{o}^*e^{jk_{0}sin(\theta)x}
	\\=|\psi_{r}|^2 + |\psi_{o}|^2 + 2|\psi_{0}^*\psi_{r}|cos(2\pi f_{x}x + \phi (x,y))
	\end{multlined}
\end{equation}

The phase getting scaled by \(xsin(\theta)\) makes sense because at \(\theta = \pi/2\), the extra distance turns into just x, and at \(\theta = 0\)
the reference wave is already in phase with the object wave, so no adjustment would be necessary. Just like the earlier section,
we recognize the cosine identity in the top of the last equation (the \(\phi\) is any extra phase that came from the \(\psi_{o}\) \(\psi_{r}\) pair not counting
the angle \(\theta\)). Specifically, \(f_{x} = \frac{sin(\theta)}{\lambda}\).

Finally, we need to factor in the reconstruction light. If it has the same magnitude and direction
as the reference light, we can represent it with the following wave function (which
later gets convolved with h(x,y,z) to render the real image):

\begin{equation}
	\begin{multlined}
	\psi = \psi_{r}e^{jk_{0}sin(\theta)x}t(x,y)
	\\=|\psi_{r}|^2\psi_{r}e^{jk_{0}sin(\theta)x} + |\psi_{0}|^2\psi_{r}e^{jk_{0}sin(\theta)x}
	\\+ \psi_{o}|\psi_{r}|^2 + \psi_{o}^*\psi_{r}^2e^{2jk_{0}sin(\theta)x}
	\end{multlined}
\end{equation}

\begin{center}
\includegraphics[width=100mm]{tupac14.png}
\end{center}

Upon closer examination, we can regard \(k_{0}sin(\theta)\) our carrier frequeny (in radians). Since \(\psi_{r}\) is a reference light, it is entirely DC, so the first term
in (69) is \(\Psi_{1}\), the carrier.
The second term in (69) is \(\Psi_{2}\) because the phasor convolves the magnitude-squared of the object wave out to the carrier frequency.
The third (and most important) term, \(\Psi_{3}\) is the only one that is not modulated by the carrier (it gets cancels out by the algebra). 
This is the only term that is visible to the human eye!

By inspection, the minimal angle \(\theta\) we need to prevent aliasing can be given by:

\begin{equation}
	k_{0}sin(\theta) \geq \frac{3B}{2}
\end{equation}

\section{Project Setup}

\begin{center}
\includegraphics[width=100mm]{tupac10.png}
\end{center}



\section{List of Materials}

\section{What we have done}

\section{Future goals}

\end{document}

